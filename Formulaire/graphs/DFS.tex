\subsection{DFS}
Equals to BFS with \textit{Stack} instead of \textit{Queue} or recursive implementation. Complexity $O(|V|+|E|)$
\lstinputlisting{graphs/dfs.java}


\subsubsection{Topological order}
Graph must be acyclic.
\begin{lstlisting}[language=C++]
void dfs(int u, deque<int> &st) {
    if (vis[u]) return;
    vis[u] = true;
    for (int v : adj[u]) dfs(u);
    st.push_front(u); // no need to reverse after
}

deque<int> topo;
for (int u=0; u<n; u++) dfs(u, topo);
\end{lstlisting}

\subsubsection{Strongly connected components}
Same \lstinline|dfs()| as toposort but takes adjacency list. 
SCCs are in toposorted order. For the reverse other, switch \lstinline|adj| and \lstinline|adjT|.
\begin{lstlisting}[language=C++]
deque<int> topo;
for (int u=0; u<n; u++) dfs(u, adj, topo);
vis.reset();
for (int u : topo) {
    if (!vis[u]) {
        deque<int> scc;
        dfs(u, adjT, scc);
    }
}
\end{lstlisting}

\subsubsection{SCC, Bridges and Articulation Points in C}
C version of SCC (shorter).
\lstinputlisting{graphs/scc1.cpp}

Bridges are edges that, when removed, increases the number of connected components. Articulation points are the same, but for vertices.
\lstinputlisting{graphs/scc2.cpp}

\subsubsection{Directed Graph to toposorted DAG}
In $O(n + m)$, with Tarjan SCC algo,
we merge the SCCs and take the resulting DAG,
(remembering their size in \lstinline|scc_size|)
which is reverse toposorted (i.e. node 0 has no outgoing edge),
ready for bottom up DP (starting with node 0 ending with node $N$)!
\lstinputlisting{graphs/sccdag.java}
