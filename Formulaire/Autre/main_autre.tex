\section{Miscellaneous}
\begin{center}
	\includegraphics[width=0.7\linewidth]{whyuno.jpg}
\end{center}
\subsection{Strings {\footnotesize \textit{untested}}}
\begin{lstlisting}

\end{lstlisting}
\subsubsection{Longest palindrome}
\begin{lstlisting}
int[] calculateAtCenters(String s) {
  int n = s.length();
  int[] L = new int[2 * n + 1];
  int i = 0, palLen = 0, k = 0;
  while(i < n) {
    if((i > palLen) && 
       (s.charAt(i - palLen - 1)==s.charAt(i))) {
      palLen += 2;
      i += 1;
      continue;
    }
    L[k++] = palLen;
    int e = k - 2 - palLen;
    boolean found = false;
    for(int j = k - 2; j > e; j--) {
      if(L[j] == j - e - 1) {
        palLen = j - e - 1;
        found = true;
        break;
      }
      L[k++] = Math.min(j - e - 1, L[j]);
    }
    if(!found) {
      i += 1;
      palLen = 1;
    }
  }
  L[k++] = palLen;
  int e = 2 * (k - n) - 3;
  for(i = k - 2; i > e; i--) {
    int d = i - e - 1;
    L[k++] = Math.min(d, L[i]);
  }
  return L;
}

String getPalindrome(String s, int[] L) {
  int max = L[0];
  int maxI = 0;
  for(int i = 1; i < L.length; i++) {
    if(L[i] > max) {
      max = L[i];
      maxI = i;
    }
  }
  int b = 0, e = 0;
  b = maxI / 2 - L[maxI] / 2;
  e = maxI / 2 + L[maxI] / 2;
  e += maxI % 2 == 0 ? 0 : 1;
  return s.substring(b, e);
}

String getPalindrome(String s)
{
	return getPalindrome(s, calculateAtCenters(s));
}
\end{lstlisting}
\subsection{The answer}
\begin{lstlisting}
int reponse() { return 42; }
\end{lstlisting}
\subsection{Occurences in a string}
KMP(s,w) returns occurences index of w in s.
\begin{lstlisting}
LinkedList<Integer> KMP(String s, String w) {
  LinkedList<Integer> matches = new  
  LinkedList<Integer>();
  int k = 0, i = -1;
  int[] t = KMPtable(w);
  do {
    i = KMP(s, w, k, t);
    if(i != -1) {
      matches.add(i);
      // change to i+len(w) disalow overlap
      k = i + 1; 
    }
  } while(i != -1 && k < s.length());
  return matches;
}
	
int KMP(String s, String w, int k, int[] t) {
  int i = 0;
  int n = s.length(), m = w.length();
  while(k + i < n) {
    if(w.charAt(i) == s.charAt(k + i)) {
      i++;
      if(i == m) {return k;}
    } else {
      k += i - t[i];
      i = t[i] > -1 ? t[i] : 0;
    }
  }
  return -1;
}

int[] KMPtable(String w) {
  int m = w.length();
  int[] t = new int[m];
  int pos = 2, cnd = 0;
  t[0] = -1;
  t[1] = 0;
  while(pos < m) {
    if(w.charAt(pos - 1) == w.charAt(cnd)) {
      t[pos++] = ++cnd;
    } else if(cnd > 0) {
      cnd = t[cnd];
    } else {
      t[pos++] = 0;
    }
  }
  return t;
}
\end{lstlisting}
\subsection{Sort algorithms {\footnotesize \textit{untested}}}
\begin{lstlisting}
int findKth(int[] A, int k, int n) {
  if(n <= 10) {
    Arrays.sort(A, 0, n);
    return A[k];
  }
  int nG = (int)Math.ceil(n / 5.0);
  int[][] group = new int[nG][];
  int[] kth = new int[nG];
  for(int i = 0; i < nG; i++) {
    if(i == nG - 1 && n % 5 != 0) {
      group[i] = Arrays.copyOfRange(A, (n/5)* 5, n);
      kth[i] = findKth(group[i], group[i].length / 2, 
                     group[i].length);
    } else {
      group[i] = Arrays.copyOfRange(A, i*5, (i+1)*5);
      kth[i] = findKth(group[i], 2, group[i].length);
    }
  }
  int M = findKth(kth, nG / 2, nG);
  int[] S = new int[n]; 
  int[] E = new int[n];
  int[] B = new int[n];
  int s = 0, e = 0, b = 0;
  for(int i = 0; i < n; i++) {
    if(A[i] < M) {
      S[s++] = A[i];
    } else if(A[i] > M) {
      B[b++] = A[i];
    } else {E[e++] = A[i];}
  }
  if(k < s) {
    return findKth(S, k, s);
  } else if(k >= s + e) {
    return findKth(B, k - s - e, b);
  }
  return M;
}

int[] countSort(int[] A, int k) { // O(n + k)
  int[] C = new int[k];
  for(int j = 0; j < A.length; j++) {
    C[A[j]]++;
  }
  for(int j = 1; j < k; j++) {
    C[j] += C[j - 1];
  }
  int[] B = new int[A.length];
  for(int j = A.length - 1; j >= 0; j--) {
    B[C[A[j]] - 1] = A[j];
    C[A[j]]--;
  }
  return B;
}
	
int[][] radixSort(int[][] nums, int k) { // O(d*(n+k))
  int n = nums.length;
  int m = nums[0].length;
  int[][] B = null;
  for(int i = m - 1; i >= 0; i--) {
    int[] C = new int[k];
    for(int j = 0; j < n; j++) {
      C[nums[j][i]]++;
    }
    for(int j = 1; j < k; j++) {
      C[j] += C[j - 1];
    }
    B = new int[n][];
    for(int j = n - 1; j >= 0; j--) {
      B[C[nums[j][i]] - 1] = nums[j];
      C[nums[j][i]] = C[nums[j][i]] - 1;
    }
    nums = B;
  }
  return nums;
}

int mergeSort(int[] a) { 
  int n = a.length;
  if(n == 1) {return 0;}
  int m = n / 2;
  int[] left = Arrays.copyOfRange(a, 0, m);
  int[] right = Arrays.copyOfRange(a, m, n);
  int inv = mergeSort(left);
  inv += mergeSort(right);
  inv += merge(left, right, a);
  return inv;
}

int merge(int[] left, int[] right, int[] a) {
  int i = 0, l = 0, r = 0, inv = 0;
  while(l < left.length && r < right.length) {
    if(left[l] <= right[r]) {
      a[i++] = left[l++];
    } else {
      inv += left.length - l;
      a[i++] = right[r++];
    }
  }
  for(int j = l; j < left.length; j++) {
    a[i++] = left[j];
  }
  for(int j = r; j < right.length; j++) {
    a[i++] = right[j];
  }
  return inv;
}

int countMinSwapsToSort(int[] a) {
  int[] b = a.clone();
  Arrays.sort(b);
  int nSwaps = 0;
  for(int i = 0; i < a.length; i++) {
    // cuidado com elementos repetidos!
    int j = Arrays.binarySearch(b, a[i]);
    if(b[i] == a[j] && i != j) {
      nSwaps++;
      swap(a, i, j);
    }
  }
  for(int i = 0; i < a.length; i++) {
    if(a[i] != b[i]) {
      nSwaps++;
    }
  }
  return nSwaps;
}

//Count (i, j):h[i] <= h[k] <= h[j], k = i+1,...,j-1.
int countVisiblePairs(int[] h) { // O(n) 
  int n = h.length;
  int[] p = new int[n];
  int[] r = new int[n];
  Stack<Integer> S = new Stack<Integer>();
  for(int i = 0; i < n; i++) {
    int c = 0;
    if(S.isEmpty()) {
      S.push(h[i]);
      p[i] = 0;
    } else {
      if(S.peek() == h[i]) {
        p[i] = p[i - 1] + 1 - r[i - 1];
      } else {
        while(!S.isEmpty() && S.peek() < h[i]) {
	   S.pop();
	   c++;
	 }
	 p[i] = c;
	 r[i] = c;
	 if(!S.isEmpty()) {
	   p[i]++;
	 }
      }
    S.push(h[i]);
    }
  }
  return sum(p);
}

void shuffle(Object[] a)
{
  int N = a.length;
  for (int i = 0; i < N; i++) {
    int r = i + (int) (Math.random() * (N-i));
    swap(a, i, r);
  }
}
\end{lstlisting}
\subsection{Huffman (compression)}
Usually used for characters, but usable with everything in which we can count occurrences. \\
Make a prefix tree we use to decode and we unstack to encode.
\begin{lstlisting}
class HuffmanNode implements Comparable<HuffmanNode>
{
  public boolean isLeaf;
  public int occurences;
  public int charIndex;
  public HuffmanNode left,right;
  public HuffmanNode(HuffmanNode left, HuffmanNode right)
  {
    this.occurences = left.occurences+right.occurences;
    this.left = left;
    this.right = right;
    isLeaf = false;
  } 
  public HuffmanNode(int charIndex, int occurences)
  {
    this.charIndex = charIndex;
    this.occurences = occurences;
    isLeaf = true;
  }
  @Override
  public int compareTo(HuffmanNode o) {
    return occurences-o.occurences;
  }
}
HuffmanNode getHuffmanTree(int[] occurences) {
  PriorityQueue<HuffmanNode> q = new PriorityQueue<HuffmanNode>();
  for(int i = 0; i < occurences.length; i++)
    q.add(new HuffmanNode(i, occurences[i]));
  while(q.size() != 1) {
    HuffmanNode right = q.poll();
    HuffmanNode left = q.poll();
    q.add(new HuffmanNode(left, right));
  }
  return q.poll();
}
void getHuffmanTable(HuffmanNode tree, BitSet[] result, BitSet current, int pos){
  if(tree.isLeaf) {
    BitSet finalBitSet = new BitSet();
    for(int i = 0; i < pos; i++)
      finalBitSet.set(i, current.get(pos-i-1));
    result[tree.charIndex] = finalBitSet;
  } else {
    BitSet leftBitSet = new BitSet();
    leftBitSet.or(current);
    leftBitSet.set(pos, false);
    getHuffmanTable(tree.left, result, leftBitSet, pos+1);
            
    BitSet rightBitSet = new BitSet();
    rightBitSet.or(current);
    rightBitSet.set(pos, true);
    getHuffmanTable(tree.right, result, rightBitSet, pos+1);
  }
}
    
//n=occurences.length
static BitSet[] getHuffmanTable(int n, HuffmanNode tree) {
  BitSet[] result = new BitSet[n];
  getHuffmanTable(tree, result, new BitSet(), 0);
  return result;
}
\end{lstlisting}
\subsection{Union Find}
\label{unionFind}
\lstinputlisting{Autre/unionfind.java}
\subsection{Fenwick Tree (RSQ solver)}
\lstinputlisting{Autre/Fenwicktree.java}
