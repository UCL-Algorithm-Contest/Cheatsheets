\subsection{Suffix array}
\includegraphics[width=\linewidth]{suffixarray.jpg}
\begin{itemize}
\item Suffix array of $algorithm$ = algorithm, gorithm, hm, ithm, lgorithm, m, orithm, rithm, thm

\item Characterized by its starting index\\
Example : Suffix array of $algorithm$: $$[0, \ 2, \ 7, \ 5, \ 1, \ 8, \ 3, \ 4, \ 6]$$
Example : Given $suf_j$ suffix beginning at index $j$, and $C(i, j, k)$ comparison result of $suf_j$ and $suf_k$ on the $2^i$ first characters.
\begin{align*}
C(i, j, k) = & \ C(i - 1, j, k) \hspace{15pt} \text{si $C(i - 1, j, k) \neq 0$} \\
             & \ C(i - 1, j + 2^{i - 1}, k + 2^{i - 1}) \hspace{15pt} \text{else}
\end{align*}

\end{itemize}
\begin{itemize}

\item Define a matrix $so$ such that:
\begin{align*}
so[i][j] = so[i][k] & \Leftrightarrow C(i, j, k) = 0 \\
so[i][j] < so[i][k] & \Leftrightarrow C(i, j, k) < 0 \\
so[i][j] > so[i][k] & \Leftrightarrow C(i, j, k) > 0 
\end{align*}

$so[i]$ is the order of sorted suffixes on the $2^i$ first characters.

\end{itemize}

\begin{itemize}

\item\textbf{Base case: }$so[0][j] = (int)s.charAt(i) $\\
Example: for $s = ccacab$ we have \\$ s[0] = [97, 97, 95, 97, 95, 96]$
\item For every $j$ we define a triplet $(l, r, j)$:
\begin{align*}
(s[i - 1][j], s[i - 1][j + 2^{i - 1}], j) & \quad \text{si $j + 2^{i - 1} < n$} \\
(s[i - 1][j], -1, j) & \quad \text{si $j + 2^{i - 1} \geq n$}
\end{align*}

\end{itemize}
\ \newline
\begin{lstlisting}
class Triple implements Comparable<Triple> {
  int l, r, index;
  public Triple(int half1, int half2, int index) {
    this.l = half1;
    this.r = half2;
    this.index = index;
  };
  public int compareTo(Triple other) {
    if(l != other.l) {
      return l - other.l;
    }
    return r - other.r;
  }
} 
\end{lstlisting}
\ \newline
\begin{lstlisting}
int[][] suffixOrder(String s) { // O(n log^2(n))
  int n = s.length();
  int lg = (int)Math.ceil((Math.log(n) / Math.log(2))) + 1;
  int[][] so = new int[lg][n];
  // initialize so[0] with character order
  for(int i = 0; i < n; i++) {
    so[0][i] = s.charAt(i);
  }
  Triple[] next = new Triple[n];
  for(int i = 1; i < lg; i++) {
    // build the next array
    for(int j = 0; j < n; j++) {
      int k = j + (1 << (i - 1));
      next[j] = new Triple(so[i - 1][j], k < n ? so[i - 1][k] : -1, j);
    }
    // sort next array
    Arrays.sort(next);
    // build so[i]
    for(int j = 0; j < n; j++) {
      if(j == 0) {
      // smallest elements gets value 0 
      so[i][next[j].index] = 0;
     } else if(next[j].compareTo(next[j - 1]) == 0) {
      // equal to previous so it gets the same value
      so[i][next[j].index] = so[i][next[j - 1].index];
     } else {
      // largest than previous so get + 1
      so[i][next[j].index] = so[i][next[j - 1].index] + 1;
     }
   }
 }
 return so;
}

//Calcule le Suffix Array pour un so donne:
int[] suffixArray(int[][] so) {
  int[] sa = new int[so[0].length];
  for(int j = 0; j < so[0].length; j++) {
    sa[so[so.length - 1][j]] = j;
  }
  return sa;
}

//Retourne le plus long prefixe commun de suf_j (le suffixe de s commencant a j = s.substr(j)) et suf_k pour un so donne:
int lcp(int[][] so, int j, int k) { // O(log(n)
  int lcp = 0;
  int n = so[0].length;
  for(int i = so.length - 1; i >= 0; i--) {
    if(j < n && k < n && so[i][j] == so[i][k]) {
      lcp += (1 << i);
      j += (1 << i);
      k += (1 << i);
    }
  }
  return lcp;
}

//Quelques exemples
String maxStrRepeatedKTimes(String s, int k) {
  int[][] so = suffixOrder(s);
  int[] SA = suffixArray(so);
  int n = s.length();
  int max = Integer.MIN_VALUE;
  int j = 0;
  for(int i = 0; i <= n - k; i++) {
    int lcp = lcp(so, SA[i], SA[i + k - 1]);
    if(lcp > max) {
      max = lcp;
      j = SA[i];
    }
  }
  return s.substring(j, j + max);
}
	
String minLexicographicRotation(String s) {
  int n = s.length();
  s += s;
  int[] SA = suffixArray(suffixOrder(s));
  int i = 0;
  while(!(0 <= SA[i] && SA[i] < n)) {
    i++;
  }
  return s.substring(SA[i], SA[i] + n);
}

class MaxLexConc implements Comparator<String> {
 public int compare(String x, String y) {
    String xy = x + y;
    String yx = y + x;
    if(xy.compareTo(yx) < 0 || 
      (xy.equals(yx) && x.length()<y.length())) {
      return 1; 
    }
    return -1; 
}
\end{lstlisting}